\documentclass[a4paper]{article}
\usepackage{graphicx}
\usepackage{subcaption}
\usepackage{amsmath}
\usepackage{amssymb}
\usepackage[utf8]{inputenc}
\usepackage[english]{babel}

\author{Mattia Lecchi (33362A)}
\title{Branch and bound algorithm for SCP with Conflict Sets\\ 
	\large Project of Operations Research Complements (A.Y. 2024/2025)}

\begin{document}
	\maketitle
	
\section{Introduction}
The project tackle a variant of the Set Covering Problem (SCP) introducing penalties if certain couples of sets are selected. Formally, let $E=\{1,...,n\}$ be a set of elements and let $\mathcal S=\{S_j\subseteq E : j \in M\}$ be the set of subsets, where $M=\{1,...,m\}$. 
The subsets composition is defined by a matrix $A \in \mathbb R^{n\times m}$:
$$
A_{ij} =
\begin{cases}
	1& \text{if } E_i \in S_j\\
	0& \text{otherwise}\\
\end{cases}
$$
The vector $c \in \mathbb R^{m}$ defines the costs on each subset, while the matrix $P \in \mathbb R^{m\times m}$ defines the penalties for each couple of subsets, i.e. $P_{ij}$ is the penalty paid if $S_i,S_j\in \mathcal S$ are both selected.
A MILP formulation for the SCP with conflict sets is the following:
\begin{align}
	\min & \sum_{i\in S} x_i c_i + \sum_{i \in S} \sum_{j \in S} y_{ij} P_{ij} & \\
	\text{s.t. } 
	& \sum_{i\in S} A_{ik} x_i \ge 1 & \forall k \in E \\
	& x_i + x_j \le 1 + y_{ij} & \forall i,j \in M \\
	& x_i \in \{0, 1\} & \forall i \in M \\
	& y_{ij} \in \{0, 1\} & \forall i,j \in M\\
\end{align}

The chosen approach is to use a branch and bound algorithm with a lagrangean relaxation used for dual bound computing.

\section{Lagrangean relaxation}

We chose to relax the covering constraints, obtaining the following lagrangean relaxation:

\begin{align}
	\min & \sum_{i\in S} x_i c_i + \sum_{i \in S} \sum_{j \in S} y_{ij} P_{ij} + \lambda_i \left(\sum_{i\in S}A_{ik} x_i - 1\right)  & \\
	\text{s.t. }
	& x_i + x_j \le 1 + y_{ij} & \forall i,j \in M \\
	& x_i \in \{0, 1\} & \forall i \in M \\
	& y_{ij} \in \{0, 1\} & \forall i,j \in M\\
\end{align}
	
	
\end{document}